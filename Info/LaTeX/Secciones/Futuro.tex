En el futuro es posible que la aplicación necesite mantenimento, o que surgan algunos bugs en el futuro. Pero por otro lado tambien puede
que quede obsoleta debido a varios motivos.

Lo que mas hay que cuidar es la parte de la API, ya que puede cambiar el formato o dejar de existir, pero esto no deberia ser tanto problema
asi ya que simplemente modificamos a la API que llamamos, y hacemso los cambios correspondientes, pasaria lo mismo si el cliente desea cambiar
de API, o utilizar otra distinta, no seria una gran modificación del codigo fuente, seria simplemente agregar funcionalidad. 

Por otro lado tambien podemos agregar mas tipos de entrada, por si desamos tambien recibir coordenadas, recibir la ciudad directamente, 
o algun otro tipo que deseemos agregar, seria algo similar que con la API, transformar esa entrada con la que ya sabemos como trabajar,
por ejemplo ICAO o IATA.