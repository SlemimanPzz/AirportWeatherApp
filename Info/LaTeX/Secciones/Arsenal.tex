\subsection{Lenguaje y Herramientas}
\label{subsec:LengHerr}

Para este proyecto se decidio que se utilizara \href{https://www.swift.org}{Swift} para la crecion de este. Ya que nos
permitira crear la aplicación para dispotivos iOS, macOs, y iPadOS, para facilitar el uso de la aplicación. Aparte de esto se
utilizara el framework de \href{https://developer.apple.com/documentation/swiftui/}{SwiftUI} proporcionado por Apple para crear la
interfaz grafica, debido a todos los elementos que nos propociona y la facilidad de uso. Y sera desarollada en en la IDE, Xcode, debido a todas las herramientas que propociona para el desarrollo para
dentro del enterno de Apple. Tambien debido a que se necesita tranformar los codigos \texttt{IATA} a \texttt{ICAO} se decidio
utilizar \href{https://realm.io/}{Realm} basado en MongoDB, el cual ya tiene un \href{https://www.mongodb.com/docs/realm/sdk/swift/#realm-swift-sdk}{Swift SDK}.
Tambien se genero una base de datos para que sea empaquetada con la applicacion y la transformacion de los codigos sea local.
Y cabe recalcar que no es necesario subir la aplicación a la App Store para dar la aplicacion a los empleados, conectando al
dispotivo de desarrollo es posible para la App a los dispotivos como una Enterprise App, solo se necesitan dar unos permisos
a esta en el dispotivo de uso.


\subsection{API}
\label{subsec:API}
Tambien para la API se decidió por \href{https://www.checkwxapi.com}{CheckWXAPI} ya que usa el codigo ICAO como base para las peticiones,
y nos da todo lo que requerimos como el clima y mas datos, que podrian resultar ser utiles. Una API sera necesaria, la expliación en la
sección \ref{sec:Requisitos}.
